\section{The Phase-2 CMS detector}
\label{section:Phase_II_CMS}

The CMS detector~\cite{Chatrchyan:2008aa} will be substantially upgraded in order to fully exploit the physics potential offered by the increase in luminosity, and to cope with the demanding operational conditions at the HL-LHC~\cite{CMSCollaboration:2015zni, Klein:2017nke, Collaboration:2283187, Collaboration:2293646, Collaboration:2283189}. The upgrade of the first level hardware trigger (L1) will allow for an increase of L1 rate and latency to about 750\unit{kHz} and 12.5\mus, respectively, and the high-level software trigger (HLT) is expected to reduce the rate by about a factor of 100 to 7.5\unit{kHz}. The entire pixel and strip tracker detectors will be replaced to increase the granularity, reduce the material budget in the tracking volume, improve the radiation hardness, and extend the geometrical coverage and provide efficient tracking up to pseudorapidities of about $|\eta|=4$. The muon system will be enhanced by upgrading the electronics of the existing cathode strip chambers (CSC), resistive plate chambers (RPC) and drift tubes (DT). New muon detectors based on improved RPC and gas electron multiplier (GEM) technologies will be installed to add redundancy, increase the geometrical coverage up to about $|\eta|=2.8$, and improve the trigger and reconstruction performance in the forward region. The barrel electromagnetic calorimeter (ECAL) will feature the upgraded front-end electronics that will be able to exploit the information from single crystals at the L1 trigger level, to accommodate trigger latency and bandwidth requirements, and to provide 160\unit{MHz} sampling allowing high precision timing capability for photons. The hadronic calorimeter (HCAL), consisting in the barrel region of brass absorber plates and plastic scintillator layers, will be read out by silicon photomultipliers (SiPMs). The endcap electromagnetic and hadron calorimeters will be replaced with a new combined sampling calorimeter (HGCal) that will provide highly-segmented spatial information in both transverse and longitudinal directions, as well as high-precision timing information. Finally, the addition of a new timing detector for minimum ionizing particles (MTD) in both barrel and endcap regions is envisaged to provide the capability for 4-dimensional reconstruction of interaction vertices that will significantly offset the CMS performance degradation due to high PU rates.

A detailed overview of the CMS detector upgrade program is presented in Ref.~\cite{CMSCollaboration:2015zni, Klein:2017nke, Collaboration:2283187, Collaboration:2293646, Collaboration:2283189, CMS:2667167, CERN-LHCC-2020-004, Collaboration:2759072}, while the expected performance of the reconstruction algorithms and pile-up mitigation with the CMS detector is summarised in Ref.~\cite{Collaboration:2650976}.


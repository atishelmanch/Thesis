\subsection{Fully-leptonic final state}
\label{sec:TwoL} 

For events to fall into the Fully-leptonic category, they must contain at least one diphoton candidate, and at least two oppositely charged leptons ($e^+ e^-$, $\mu^+ \mu^-$, $e^{\pm} \mu^{\mp}$)
passing the electron and muon object selections described in Section \ref{sec:Phase_II_ObjectSel}. 

In order to save events with two leptonically decaying W bosons, events fall into the fully-leptonic category if they satisfy the selections listed in Table \ref{tab:FLSelections_Phase_II}, where $\Delta{R(l,l)}$ is the $\Delta{R}$ between two leptons, $m_{ll}$ is the mass of dilepton system and $m_{e\gamma}$ is the invariant mass of the leading electron and the leading photon in the events that have at least one electron. 

\begin{table}[!h]
    \begin{center}
        \begin{tabular}{c|c}
        Variable & Selection \\ \hline
        $\Delta{R(l,l)}$ & $> 0.4$ \\
        \pt of leading lepton & $> 20\GeV $\\
        \pt of subleading lepton & $> 10\GeV$ \\
        $E_T^{miss}$ & $> 20\GeV$ \\
        $p_T^{\gamma\gamma} $ & $> 91 \GeV$ \\
        $m_{ll}$ & $<80 \GeV$ or $>100 \GeV $ \\
        number of medium-tagged b-jets & $ = 0 $ \\
        $|m_{e\gamma} - m_{z}|$ & $ > 5\GeV$ \\
        %Third lepton veto (No third lepton with \pt) & $> 10 \GeV$ \\
        \end{tabular}
    \end{center}
    \caption{
      Selection criteria of the Fully-leptonic Channel.
    }
    \label{tab:FLSelections_Phase_II}
\end{table}
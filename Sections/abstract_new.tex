One of the few remaining parameters of the standard model (SM) of particle physics that has not yet been measured is the Higgs self-coupling. A measurement of this parameter would serve as a fundamental test of the SM, as its value has a direct effect on the shape of the Higgs potential, and any deviation of the measured Higgs self-coupling value from the SM prediction would have profound implications on physicists' collective understanding of the universe. This coupling is directly accessible via Higgs pair production, the production of two Higgs bosons in a single process, which is actively studied at the CMS experiment at the LHC. 

This thesis chronologically describes efforts by the CMS collaboration to probe the Higgs via pair production in the two W boson two photon channel, while optimizing its sensitivity by improving detection and analysis methods. In the ``past" portion, the first search by the CMS experiment for Higgs pair production in the WW$\gamma\gamma$ final state, performed using data collected at the LHC from 2016-2018 (Run 2), is presented. The dataset corresponds to a center-of-mass energy of 13 TeV, and an integrated luminosity of 138 fb$^{-1}$. The search results in an observed (expected) 95\% confidence level (CL) upper limit on Higgs pair production in the gluon-fusion channel of 3.0 (1.7) pb, corresponding to about 97 (53) times the standard model prediction. Additionally, the analysis makes use of an Effective field theory (EFT) framework to obtain an observed (expected) constraint on the strength of the Higgs self-coupling of -25.9 (-14.5) to 24.1 (18.4) times its SM value. Within the same framework, constraints are also placed on purely beyond the standard model (BSM) scenarios. An observed (expected) constraint on the strength of the direct coupling of two Higgs bosons to two top quarks of -2.4 (-1.7) to 2.9 (2.2) is obtained at a 95\% CL, and 95\% CL upper limits are placed on twenty EFT benchmark scenarios ranging from 1.7 - 6.2 (1.0 - 3.9) pb. This thesis also includes a description of the experimental setup necessary for this analysis, comprising the LHC and CMS. A detailed description of the CMS Electromagnetic Calorimeter, crucial for the detection of these di-Higgs final states, is included. In the ``present" portion, a description of the ongoing optimizations and operational plans of the CMS ECAL for LHC Run 3 is presented. Additionally, the first Phase-II projection of Higgs pair production in the WW$\gamma\gamma$ and $\tau\tau\gamma\gamma$ final states at the future HL-LHC with the Phase-II upgraded CMS detector is included, where an expected significance of 0.22 $\sigma$ is extracted.
Through the study of particle physics, human beings are able to probe the fundamental building blocks of the very universe they inhabit. This field is rich in expertise. Firstly, particle physicists seeking to answer fundamental questions about the universe determine the experimental setups, namely the types of particle accelerators and detectors, that are required in order to attempt to answer these questions. Incredible efforts are then required of engineers and technicians to build these accelerators and detectors. Multi-year long data-taking periods then commence, warranting undying dedication from a multitude of accelerator and detector operations teams. The unprecedented amounts of data recorded by these particle detectors, and their associated simulations, then require state of the art computational tools and resources to be processed and analyzed in a reasonable time frame. Finally, novel statistical interpretations must then be invoked in order to infer results from the recorded data, in an attempt to answer the questions with which physicists began. This monumental effort requires expertise from scientists around the world, where CERN (Organisation Europ\'eenne pour la Recherche Nucl\'eaire) is one of the main sites where these collaborators come together. 

In 2012, the experimental discovery of the Higgs boson by the CMS (Compact Muon Solenoid) and ATLAS experiments at the CERN LHC (Large Hadron Collider) \cite{Aad:2012tfa,Chatrchyan:2012ufa,Chatrchyan:2013lba} marked a historic achievement for humanity. Physicists had finally experimentally verified the existence of the Higgs boson, a particle first theorized in the 1960s, and one that is central to the current framework of particle physics: The Standard Model (SM).

The SM is a quantum field theory describing the fundamental forces of nature and its constituent particles. It accounts for the electromagnetic, weak, and strong forces and includes particles classified as quarks, leptons, and bosons. The SM is consistent with the vast majority of experimental observations in particle physics, and continues to be widely used to this day when interpreting data. Additionally, it is often used as a basis from which extensions are made to new theories. In the SM, observable particles arise as excitations of their corresponding quantum fields. The Higgs field adds a potential energy term to the SM Lagrangian, and due to its shape and the requirement of a vacuum energy value, its mathematically described symmetry is spontaneously broken. This process is known as electroweak symmetry breaking, and is a crucial mechanism described by the SM as it explains the origin of mass for massive particles which interact with the Higgs field.

In the post Higgs-discovery era, physicists are aiming to further understand and characterize the Higgs boson, and by extension the electroweak symmetry breaking mechanism. A fundamental test of the SM is to measure the coupling strengths of the Higgs boson to massive SM particles, where many couplings have now been measured with precisions down to around 10\% with respect to their SM predicted values. A coupling of particular interest which has not yet been precisely measured is the Higgs self-coupling. This coupling has a direct effect on the shape of the Higgs potential, and its magnitude is explicitly predicted by the SM. Based on this predicted value and the current measurements of the top quark mass and Higgs boson mass, it is predicted that the Higgs vacuum energy currently sits at a metastable minimum. This would mean there is a non-zero probability that the minimum of the Higgs potential can tunnel to a lower minimum, which would change the laws of physics as we know them. An experimental measurement of the Higgs self-coupling could have profound implications on physicists' understanding of the universe: It may confirm the predictions of the SM, or it may refute the SM entirely and prompt physicists to make sense of this in a way that is in agreement with all previous observations. 

Through the study of Higgs pair production, the production of two Higgs bosons in a single process, the Higgs self-coupling is directly accessible. Additionally, through the use of an Effective Field Theory (EFT) framework, a wide model-independent search for physics beyond the standard model (BSM) through searches of Higgs pair production can be performed. Due to destructive interference between its two leading order (LO) processes, non-resonant Higgs pair production has a low production cross section, meaning that the combination of many di-Higgs final states will be necessary in order to maximize the chances for observation, and obtain as precise a measurement as possible of the Higgs self-coupling. In order to search for this relatively small signal and include a new search channel, a search for Higgs pair production by the CMS experiment in the WW$\gamma\gamma$ channel was performed. This channel benefits from the sensitive H$\rightarrow\gamma\gamma$ process which provides a narrow, distinguishable signature. The H$\rightarrow$WW leg of the decay contributes a relatively large branching ratio among Higgs boson decay processes of about 22\%, increasing the expected yields of the di-Higgs process. Because the W boson can decay both leptonically and hadronically, the H$\rightarrow$WW and by extension the HH$\rightarrow$WW$\gamma\gamma$ process has three possible final states: The Fully-Hadronic, Semi-Leptonic, and Fully-Leptonic final states, corresponding to 0, 1, and 2 leptonically decaying W-bosons respectively.

During Run 2 of the LHC (2016-18), about 138 fb$^{-1}$ of data was recorded by the CMS detector from proton-proton collisions at a center-of-mass energy of $\sqrt{s}=13$ TeV. Using this dataset, the first search for Higgs pair production in the WW$\gamma\gamma$ final state by the CMS experiment was performed. The CMS ECAL (Electromagnetic Calorimeter) was vital for this analysis, as it is the sole sub-detector of CMS that can directly detect photons which leave a trace of the sensitive H$\rightarrow\gamma\gamma$ signature in all three final states of this di-Higgs channel. It is additionally crucial for the detection of electrons, present in two of the three WW$\gamma\gamma$ final states. 

In order to potentially improve the sensitivity of this analysis and similar analyses using the CMS dataset to be collected during Run 3 of the LHC beginning in 2022, existing CMS ECAL features have been optimized and new features have been investigated. During the commissioning period of the CMS ECAL for Run 3, the performance of the CMS ECAL operations teams will be crucial for the detector's successful operation and its commissioning of these features.

Additionally, it is important to estimate the expected sensitivity of this analysis and similar analyses using data to be collected by the upgraded CMS detector during the running period of the future High Luminosity LHC, in order to gauge the prospects of these future physics analyses. This is done through projection studies, the first of which in the WW$\gamma\gamma$ and $\tau\tau\gamma\gamma$ di-Higgs final states has been performed by the CMS collaboration.

This thesis is organized as follows: Chapter \ref{chapter:TheoreticalBackground} will describe the theoretical background which comprises our current understanding of particle physics, and inherently motivates experimental analyses. Chapter \ref{chapter:Experimental_Setup} will describe the experimental setup: The Large Hadron Collider and the CMS detector. Chapter \ref{chapter:HHWWyy} makes up the ``Past" portion of this thesis, and will describe the first search by the CMS collaboration for Higgs pair production in the two W-boson and two-photon final state. Chapter \ref{chapter:ECAL_Run3} makes up the ``Present" portion of this thesis, and will describe the ongoing efforts to optimize the CMS Electromagnetic Calorimeter for Run 3, a crucial subdetector for detecting photon and electron signatures of the HH$\rightarrow$WW$\gamma\gamma$ process. This chapter will also include a description of the roles of the various CMS ECAL operations teams. Chapter \ref{chapter:Phase_II_HH} makes up the ``Future" portion of this thesis, and will describe the first sensitivity projection of the search for Higgs pair production in the WW$\gamma\gamma$ and $\tau\tau\gamma\gamma$ final states at the future High Luminosity LHC with the upgraded CMS detector. Finally, Chapter \ref{chapter:summary} will summarize the full content of the thesis, and highlight the importance of making the most of the data taken in the past while simultaneously optimizing the quality of the data taken during the present, and investigating the potential prospects for physics analysis using data to be taken in the future. 
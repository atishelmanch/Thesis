The Standard Model of particle physics, while widely successful and consistent with the majority of physical observations, does not account for all observed phenomena. Through the ongoing physics studies by the experiments at the CERN LHC, physicists continue to attempt to answer questions about the fundamental nature of the universe, and test the standard model by searching for and measuring new processes. Part of this analysis program includes the search for Higgs pair production, as it provides direct access to one of the unmeasured parameters of the SM: The Higgs self-coupling. As Higgs pair production is a rare process, it is vital to include as many di-Higgs decay channels as possible to maximize the chances for discovery at the future HL-LHC, and an eventual precise measurement of the Higgs self-coupling.

To this end, the first search for Higgs pair production in the WW$\gamma\gamma$ final state has been performed by the CMS experiment. The analysis made use of data collected by the CMS detector from 2016-2018, corresponding to an integrated luminosity of 138 \unit{fb}$^{-1}$, from proton-proton collisions at a center-of-mass energy of 13 TeV. The analysis resulted in an observed (expected) 95\% $CL_{s}$ upper limit on the di-Higgs production cross-section of 3.0 (1.7) pb, corresponding to about 97 (53) times the standard model prediction. In addition to the search for the SM interpretation of di-Higgs, scans of modified SM and purely BSM coupling parameters in an EFT framework result in an observed (expected) constraint on the Higgs self-coupling of -25.9 (-14.5) to 24.1 (18.4) times its standard model value, and a constraint on the magnitude of the direct coupling of two top quarks to two Higgs bosons of -2.4 (-1.7) to 2.9 (2.2) at a 95\% $CL_{s}$. Additionally, observed (expected) 95\% $CL_{s}$ upper limits are placed on twenty EFT benchmark scenarios ranging from 1.7 - 6.2 (1.0 - 3.9) pb.

As it is not yet possible to experimentally observe Higgs pair production with the Run 2 dataset due to its low production cross section, the optimization of the CMS detector for LHC Run 3 will be crucial for improving the detector's efficiency at triggering on Higgs pair production events. For this analysis and similar analyses, the use of the CMS electromagnetic calorimeter is of vital importance for the detection and accurate reconstruction of photons. In order to maintain and improve the sensitivity of this and other analyses using photons and electrons, during LS2, several optimizations of the CMS ECAL were investigated. This included updating the parameters of the on-detector energy reconstruction algorithm to account for updated ECAL signal signature shapes, and the preliminary investigation of a new feature to be used for out-of-time signal tagging. The re-optimization of the energy reconstruction parameters is expected to improve the accuracy of ECAL energy measurements used as input to the CMS level-1 trigger, which can potentially increase the detector's efficiency at triggering on di-Higgs events. Additionally, during LS2, the existence of a second amplitude filter in the on-detector ECAL electronics was discovered, which combines with the standard amplitude filter to form ``double weights". Its existence and functionality was proven through tests at a test electronics setup, and at the CMS ECAL itself. Its functionality was then added to the CMS ECAL emulator in order to test its impact on previously taken CMS data. Since then, emulator and data studies have been performed in order to test the potential removal of anomalous signals called ECAL spikes. Initial studies show that double weights may be able to remove some ECAL spikes via out-of-time tagging, with a minimal effect on ECAL signals. The potential gain of this feature may include the removal of ECAL spikes at L1, allowing CMS to trigger on more physics events including di-Higgs events, which may improve the sensitivity of di-Higgs analyses using the LHC Run 3 dataset. The possibility of using double weights to tag out-of-time physics signatures may also be investigated in the future, which may potentially improve the sensitivity of analyses targeting BSM scenarios, including those which include the Higgs boson as a bridge to BSM physics.

As LS2 has ended, so has the commissioning of the CMS detector for LHC Run 3. During this commissioning period, the CMS ECAL operations teams all performed well and are in good shape for operating the CMS ECAL and ES for LHC Run 3. During commissioning, the ECAL and preshower detectors were turned on, various tests on electronics were performed, and the commissioning of new features including double weights was performed. 

While the past Run 2 dataset has been analyzed, and the CMS ECAL is continuing to optimize its features for the Run 3 dataset, the first Phase II projection of the search for Higgs pair production in the \wwgg and \ttgg final states with the Phase-II CMS detector has been performed. The analysis strategy makes use of the Run 2 analysis strategy, including the use of deep neural networks in order to improve expected sensitivity, resulting in an expected significance of 0.22 $\sigma$.

During LHC Run 3 and at the future HL-LHC, the multitude of expertise's in the field of particle physics will continue their endeavors: Engineers and technicians will continue to maintain the LHC accelerator and CMS detector, computing experts and maintainers will continue to keep the CERN offline computing resources alive, and particle physicists will continue to develop novel analysis techniques in order to perform statistical interpretations of the newly collected datasets in order to continue answering questions about the fundamental nature of the universe. 